\documentclass[a4paper]{article}
\usepackage[utf8]{inputenc}
\usepackage[spanish]{babel}
\usepackage[margin=1.25in]{geometry}
\usepackage{mathptmx}
\usepackage{hyperref}

\title{
	\fontfamily{phv}\selectfont
	\textbf{Diseño e implementación de un servicio de búsqueda multi-hilo}
}
\author{
	\fontfamily{phv}\selectfont
	Ángel Pérez Porras \\
	\fontfamily{phv}\selectfont
	E.S. de Informática, Universidad de Castilla-La Mancha \\ \\
	\fontfamily{phv}\selectfont
	\hyperlink{mailto:angel.perez7@alu.uclm.es}{angel.perez7@alu.uclm.es}
}
\date{\fontfamily{phv}\selectfont Junio de 2021}

\begin{document}
	\maketitle
	
	\begin{abstract}
		Para la tercera práctica de la asignatura de Sistemas Operativos II, se solicita el diseño e implementación de un programa en el lenguaje C++ siguiendo el paradigma productor-consumidor en el que se presta un servicio de búsqueda textual a clientes que lo solicitan de forma dinámica. En esta propuesta se implementan varios servicios que, mediante el paso de mensajes y utilizando características de concurrencia modernas de la librería estándar de C++, simulan peticiones de dichos servicios por parte de clientes que se crean continuamente. La solución se materializa en la producción de dos artefactos \textit{software}; a saber, un prototipo escrito en el lenguaje de programación Python (\texttt{protofind.py}) y una implementación final del programa \texttt{mtfind2(1))} (haciendo alusión al producto de la práctica anterior).
	\end{abstract}

	\begin{section}
				
	\end{section}
\end{document}
